\documentclass{beamer}
\usetheme{default}
\title[Intro to Git]{Intro to Git}
\subtitle[short]{Just the Basics}
\author[Aaron Lee]{Aaron Lee (wwkeyboard@gmail.com)}
\begin{document}

\begin{frame}[plain]
  \titlepage
\end{frame}

\begin{frame}{How do we share code?}
  \begin{center}
    \begin{itemize}
      \item email
      \item shared directory
      \item projector or skype
    \end{itemize}
  \end{center}
\end{frame}

\begin{frame}{What is a diff}
  \begin{center}
    \includegraphics[width=0.7\textwidth]{diff.png}
  \end{center}
\end{frame}

\begin{frame}{Introducing Git\!}
  \begin{center}
    \includegraphics[width=0.7\textwidth]{git_homepage.png}
  \end{center}
\end{frame}

\begin{frame}{Installing Git}
  \begin{center}
    \begin{itemize}
      \item Any recent version will do
      \item There are some GUIs(including RStudio!), but most rely on
        the CLI tools.
      \item Check your package manager(apt, yum, homebrew)
    \end{itemize}
  \end{center}
\end{frame}

\begin{frame}{Global Configuration}
  \begin{center}
    % \begin{itemize}
    % \item git config \-\-global user.name "Aaron Lee"
    % \item git config \-\-global user.email "your_email\@example.com"
    % \end{itemize}
    \includegraphics[width=0.7\textwidth]{global_config.png}
  \end{center}
\end{frame}

\begin{frame}{Gotchas and good-to-knows}
  \begin{center}
    \begin{itemize}
      \item A git directory is self contained
      \item Each commit has a unique ID (that can be referenced
        anywhere)
      \item The first line of a commit message should be descriptive
      \item Add files and directories you don't want tracked to
        .gitignore
    \end{itemize}
  \end{center}
\end{frame}

\begin{frame}{Setup a Repository}
  \begin{center}
    \begin{itemize}
      \item git init
      \item git add .
      \item git status
      \item git commit -m "Something descriptive"
      \item git status
      \item git log
    \end{itemize}
  \end{center}
\end{frame}

\begin{frame}{Use Another's Repository}
  \begin{center}
    \begin{itemize}
      \item git clone git://github.com/hadley/devtools.git
      \item cd devtools
      \item ls -la
      \item git log
      \item git blame R/news.r
      \end{itemize}
  \end{center}
\end{frame}

\begin{frame}{Branching}
  \begin{center}
    \begin{itemize}
      \item (start with an initalized repo)
      \item git branch my\_feature
      \item git checkout my\_feature
      \item git branch -d my\_feature
      \item git checkout -b another\_feature
      \item (make some changes, commit them)
      \item git checkout master
      \item git merge master my\_feature
    \end{itemize}
  \end{center}
\end{frame}

\begin{frame}{Terms}
  \begin{center}
    \begin{itemize}
      \item commit - \$ git add - staged - \$ git commit
      \item branch - \$ git branch
      \item repository
      \item \$ git clone - \$ git pull - \$ git push
    \end{itemize}
  \end{center}
\end{frame}

\begin{frame}{Advanced Features}
  \begin{center}
    \begin{itemize}
    \item git bundle
    \item run a local server
    \item git bisect
    \item GUI tools (RStudio, gitk, github)
    \end{itemize}
  \end{center}
\end{frame}

\begin{frame}{Further Reading}
  \begin{center}
    \begin{itemize}
      \item http://gitready.com/
      \item http://www-cs-students.stanford.edu/~blynn/gitmagic/
      \item https://help.github.com/
      \item http://git-scm.com/book
    \end{itemize}
  \end{center}
\end{frame}
\end{document}
